Pour calculer des nombres premiers inférieurs à \(k\), on fait une version séquentielle basée sur la
technique du crible d'Eratosthène, puis envisager et réaliser différentes formes de parallélisation de
cet algorithme. (\iCode{PrimeSearch.java})
Une comparaison deux différentes versions :
\begin{itemize}
    \item
          \iCode{SequentialSearch}: L'exécution séquentielle est monothread, l'efficacité de l'exécution du code est
          très faible.
    \item
          \iCode{ParallelSearch}: Plusieurs threads exécutés en parallèle ensemble, l'efficacité est relativement
          élevée.
\end{itemize}
